\documentclass[a4paper,12pt]{article}
\usepackage[utf8]{inputenc}
\usepackage[T1]{fontenc}
\usepackage[defaultsans]{opensans}

\DeclareRobustCommand\ebseries{\fontseries{eb}\selectfont}
\DeclareRobustCommand\sbseries{\fontseries{sb}\selectfont}
\DeclareRobustCommand\ltseries{\fontseries{l}\selectfont}
\DeclareRobustCommand\clseries{\fontseries{cl}\selectfont}

\DeclareTextFontCommand{\texteb}{\ebseries}
\DeclareTextFontCommand{\textsb}{\sbseries}
\DeclareTextFontCommand{\textlt}{\ltseries}
\DeclareTextFontCommand{\textcl}{\clseries}

\begin{document}

\section{Introduction}

\section{blablabla}

\section{Conception ...}
\subsection{Le concept d'un module}
\subsection{Les éléments côté client}
%parler du MVC (séparation des vues de la donnée)%
\subsubsection{Présentation de la vue générale}
\subsubsection{ Les composants} 
\subsubsection{ L'arbre des composants}
\subsubsection{ Les fonctions}
\subsubsection{ Les actions}
\subsubsection{ Les variables}
\subsubsection{ La configuration}

\subsection{Les éléments côté serveur}
\subsubsection{Les resources}
\subsubsection{Les opérations}
\subsubsection{Les données Externes}%base de données + process%

\subsection{Overview}

\subsection{Conception du portail applicatif}%penser à definir la
   %notion de SPA, s'assurer que module est défini%
\subsubsection{Description}
%décrir le concept d'installation du module %
\subsubsection{Description}

\section{Implémentation ...}
\section{conclusion}


\end{document} 
